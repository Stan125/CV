%%%%%%%%%%%%%%%%%%%%%%%%%%%%%%%%%%%%%%%%%
% Plasmati Graduate CV
% LaTeX Template
% Version 1.0 (24/3/13)
%
% This template has been downloaded from:
% http://www.LaTeXTemplates.com
%
% Original author:
% Alessandro Plasmati (alessandro.plasmati@gmail.com)
%
% License:
% CC BY-NC-SA 3.0 (http://creativecommons.org/licenses/by-nc-sa/3.0/)
%
% Important note:
% This template needs to be compiled with XeLaTeX.
% The main document font is called Fontin and can be downloaded for free
% from here: http://www.exljbris.com/fontin.html
%
%%%%%%%%%%%%%%%%%%%%%%%%%%%%%%%%%%%%%%%%%

%----------------------------------------------------------------------------------------
%	PACKAGES AND OTHER DOCUMENT CONFIGURATIONS
%----------------------------------------------------------------------------------------

\documentclass[a4paper,10pt]{article} % Default font size and paper size

\usepackage{fontspec} % For loading fonts
\defaultfontfeatures{Mapping=tex-text}
\setmainfont{Calibri} % Main document font

\usepackage{xunicode,xltxtra,url,parskip} % Formatting packages

\usepackage[usenames,dvipsnames]{xcolor} % Required for specifying custom colors

\usepackage{fullpage} % Margin formatting of the A4 page, an alternative to layaureo can be \usepackage{fullpage}
%To reduce the height of the top margin uncomment: 
%\addtolength{\voffset}{-1.3cm}

\usepackage{hyperref} % Required for adding links	and customizing them
\definecolor{linkcolour}{rgb}{0,0.2,0.6} % Link color
\hypersetup{colorlinks,breaklinks,urlcolor=linkcolour,linkcolor=linkcolour} % Set link colors throughout the document

\usepackage{titlesec} % Used to customize the \section command
\titleformat{\section}{\Large\scshape\raggedright}{}{0em}{}[\titlerule] % Text formatting of sections
\titlespacing{\section}{0pt}{3pt}{3pt} % Spacing around sections

\begin{document}

\pagestyle{empty} % Removes page numbering

\font\fb=''Calibri'' % Change the font of the \LaTeX command under the skills section

%----------------------------------------------------------------------------------------
%	NAME AND CONTACT INFORMATION
%----------------------------------------------------------------------------------------

\par{\centering{\Huge Stanislaus \textsc{Stadlmann}}\bigskip\par} % Your name

\section{Persönliche Daten}

\begin{tabular}{rl}
\textsc{Geburtsort und -datum} & Wien, Österreich  | 14.08.1992 \\
\textsc{Addresse:} & Bertheaustr. 1, D-37075 Göttingen \\
\textsc{Telefon:} & +49 173 692 5237\\
\textsc{Email:} & \href{mailto:stanislaus@stadlmann.cm}{stanislaus@stadlmann.cm} \\
\textsc{GitHub Seite:} & \href{https://github.com/Stan125}{github.com/Stan125}
\end{tabular}

%----------------------------------------------------------------------------------------
%	EDUCATION
%----------------------------------------------------------------------------------------

\section{Bildung}

%-----------------------------------------------

\begin{tabular}{rl}
\textsc{Aktuell}& M.Sc. in Applied Statistics , \textbf{Georg-August-Universität} Göttingen\\
\textsc{Okt 2015}& \emph{{{\O}}-Note: 1,7 | Schwerpunkte: Wirtschaftswissenschaften}\\
&\\


%-----------------------------------------------


\textsc{Mrz 2015}& B.A. in Economics, \textbf{Georg-August-Universität} Göttingen\\
\textsc{Okt 2011}& \emph{{{\O}}-Note: 2,0 | Schwerpunkte: Statistik / Ökonometrie}\\
&\footnotesize{Bachelor-Arbeit: "Modeling the Relationship between
Unemployment Rate and}\\
&\footnotesize{Population Density including other
Socioeconomic Factors in Districts of
Germany}\\
& \footnotesize{using Semi-Parametric Regression"} \\
&\\

%------------------------------------------------

\textsc{Aug 2013} & Auslandssemester, \textbf{University of Ireland}, Galway\\
\textsc{Dez 2013}& \emph{Schwerpunkte: Wirtschaft Irlands, Geld \& Bankenwesen}\\
&\\

\textsc{Jun 2011} & Gymnasiale Oberstufe / Abitur, \textbf{Altkönigschule}, Kronberg im Taunus\\
\textsc{Aug 2008}& \emph{{\O}-Note: 2,0 | Leistungskurse: Polititk und Wirtschaft, Chemie}\\

\end{tabular}

%----------------------------------------------------------------------------------------
%	WORK EXPERIENCE 
%----------------------------------------------------------------------------------------

\section{Praktische Erfahrung}

\begin{tabular}{r|p{11cm}}

%------------------------------------------------

\emph{Mrz 2016} & Praktikant \& Werkstudent bei \textbf{STATWORX}, Frankfurt\\
\textsc{Mai 2015} & \emph{Data Science - Team} \\
& \footnotesize{
$\bullet$ Erstellung und Präsentieren von umfangreichen Lern-Inhalten zum Thema Statistik und R

$\bullet$ Abwicklung von mehreren Projekten (bis zu 3 Monate) bei Firmen- sowie Privatkunden zum Thema Data Science, Statistik und Modellierung}\\
\multicolumn{2}{c}{} \\


%------------------------------------------------

\emph{Apr 2015} & Praktikant bei \textbf{Landesbank Baden-Württemberg}, Stuttgart\\
\textsc{Nov 2014} & \emph{Credit Research} \\
& \footnotesize{
$\bullet$ Mitwirkung bei der Erstellung und Pflege eines Kreditwürdigkeitsmodells

$\bullet$ Erstellung von qualitativen und quantitativen Publikationen zum Thema Staatsanleihen}\\
\multicolumn{2}{c}{} \\

%------------------------------------------------

\textsc{Okt 2014} & Tutor für Statistik an der \textbf{Georg-August-Universität}, Göttingen\\
\textsc{Okt 2012} & \emph{Lehrstuhl für Statistik und Ökonometrie}\\
& \footnotesize{
$\bullet$ Halten von mehrwöchigen Tutorien zu Statistik und dem Programm R

$\bullet$ Vor- und Aufbereiten von Lerninhalten

$\bullet$ Halten von einer Statistik-Großübung vor ca. 200 Studenten}\\
\multicolumn{2}{c}{} \\

%------------------------------------------------

\textsc{Aug 2015} & Lehr-Person bei \textbf{Ländliche Erwachsenen-Bildung}, Göttingen\\
\textsc{ \& Aug 2014}& $\bullet$ \footnotesize{Organisieren, Vorbereiten und Vortragen eines fünftägigen Seminars zum Thema} \\

\textsc{\& Aug 2013}& "\footnotesize{\emph{Statistik im Qualitätsmanagement}"}\\
\multicolumn{2}{c}{} \\

%------------------------------------------------

\end{tabular}


%----------------------------------------------------------------------------------------
%	LANGUAGES
%----------------------------------------------------------------------------------------

%\section{Sprachen}

%\begin{tabular}{rlrl}
%\textsc{Deutsch:} & Muttersprache & \textsc{Französisch:} & Basiskenntnisse\\
%\textsc{Englisch:} & Fließend &\textsc{Spanisch:}& Basiskenntnisse\\
%\end{tabular}

%----------------------------------------------------------------------------------------
%	COMPUTER SKILLS 
%----------------------------------------------------------------------------------------
\section{Programm-Kenntnisse}
\begin{tabular}{rl}
Basis: & Python, Stata, eViews, Minitab, VBA\\

Fortgeschritten: & SPSS, Excel, Word, PowerPoint, LaTeX\\

Experte: & R
\end{tabular}

%----------------------------------------------------------------------------------------
%	STATISTICS SKILLS
%----------------------------------------------------------------------------------------
%\section{Kenntnisse in der Statistik}
%\begin{tabular}{rl}
%Basis: & Multivariate Analysen: \\

%Fortgeschritten: & SPSS, Excel, Word, PowerPoint, LaTeX\\

%Experte: & R
%\end{tabular}

\end{document}
